
\documentclass{article}

\makeatletter
\makeatother


\usepackage{tikz,pgflibraryshapes}
\usetikzlibrary{arrows, calc, decorations.pathmorphing,plotmarks}

\usepackage[psfixbb,graphics,tightpage,active]{preview}



\PreviewEnvironment{tikzpicture}
\newlength{\imagewidth}
\newlength{\imagescale}

\begin{document}

\usetikzlibrary{calc,arrows}
% constants
\pgfmathsetmacro\xmin{0};
\pgfmathsetmacro\xmax{10};
\pgfmathsetmacro\ymin{0};
\pgfmathsetmacro\ymax{10};

% style
\tikzset{boundary/.style={draw,color=black,line width=2pt}}
\tikzset{levelset/.style={draw,color=blue,line width=2pt}}
\tikzset{axis/.style={line width=1, fill=gray, draw=gray,-triangle 45,postaction={draw, line width=1, shorten >=7, -}}}
\tikzset{arrow/.style={draw,color=black!50!white,line width=1pt}}
\tikzset{empty node/.style={inner sep=0,outer sep=0}}

\newpage
\begin{center}
	\begin{tikzpicture}
		\footnotesize
		\node[inner sep=0pt] (d) at (3.4,2.6)
		{\includegraphics[width=1\textwidth, trim=50 320 40 0, clip]{../Figures/legend.pdf}};

		\node[inner sep=0pt] (a) at (0,0)
		{\includegraphics[width=.49\textwidth]{../Figures/semicircle_uniform_1000_all.pdf}};
		\node[inner sep=0pt] at (-1.8, 1.6) {(a)};

		\node[inner sep=0pt] (b) at (6.4,0)
		{\includegraphics[width=.49\textwidth]{../Figures/semicircle_uniform_5000_all.pdf}};
		\node[inner sep=0pt] at (4.6, 1.6) {(b)};

		\node[inner sep=0pt] (c) at (0,-4.6)
		{\includegraphics[width=.49\textwidth]{../Figures/semicircle_parabola_1000_all.pdf}};
		\node[inner sep=0pt] at (-1.8, -3) {(c)};

		\node[inner sep=0pt] (d) at (6.4,-4.6)
		{\includegraphics[width=.49\textwidth]{../Figures/semicircle_parabola_5000_all.pdf}};
		\node[inner sep=0pt] at (4.6, -3) {(d)};

		\node[inner sep=0pt] (e) at (0,-9.2)
		{\includegraphics[width=.49\textwidth]{../Figures/semicircle_triangle_1000_all.pdf}};
		\node[inner sep=0pt] at (-1.8, -7.6) {(e)};

		\node[inner sep=0pt] (f) at (6.4,-9.2)
		{\includegraphics[width=.49\textwidth]{../Figures/semicircle_triangle_5000_all.pdf}};
		\node[inner sep=0pt] at (4.6, -7.6) {(f)};
	\end{tikzpicture}
\end{center}
\end{document} 

